~
Um fabricante de plásticos planeja criar um novo produto a partir de misturas de 4 compostos químicos que chamaremos simplesmente de 1, 2, 3 e 4. Estes compostos são feitos de três elementos A, B e C (e outros elementos irrelevantes). [...]

\begin{table}[H]
    \centering
    \begin{tabular}{ccccc}
        \toprule
        Composto Químico & 1 & 2 & 3 & 4 \\
        \midrule
        Porcentagem de A & 30 & 20 & 40 & 20 \\
        Porcentagem de B & 20 & 60 & 30 & 40 \\
        Porcentagem de C & 40 & 15 & 25 & 30 \\
        \midrule
        Custo por Kg & 20 & 30 & 20 & 15 \\
        \bottomrule
    \end{tabular}
\end{table}

O novo produto deve ser constituído de exatamente 20\% do elemento A, pelo menos 30\% do elemento B e pelo menos 20\% do elemento C. Devido a efeitos colaterais dos compostos 1 e 2, eles não podem exceder 30\% e 40\% da composição do novo produto, respectivamente. Nesta questão você deve projetar um programa linear para encontrar o modo mais barato de fazer a mistura para um quilo do produto.

\itemdsep
\def\eA{{\text{A}}}
\def\eB{{\text{B}}}
\def\eC{{\text{C}}}
\def\eX{{\text{X}}}

Considere $p_{\eX,i}$ como a proporção do elemento X no composto $i$ e como $r_i$ a proporção do composto $i$ na mistura. Então, a porcentagem $p_\eX$ do elemento X na mistura final é dada por:

\[
    p_\eX = \sum_{i} p_{\eX,i} r_i
        = p_{\eX,1} r_1 + p_{\eX,2} r_2 + p_{\eX,3} r_3 + p_{\eX,4} r_4
\]

Com isso, temos que as restrições da mistura devem ser dadas por
\begin{align*}
    p_\eA &= p_{\eA,1} r_1 + p_{\eA,2} r_2 + p_{\eA,3} r_3 + p_{\eA,4} r_4 = 20\% \\
    p_\eB &= p_{\eB,1} r_1 + p_{\eB,2} r_2 + p_{\eB,3} r_3 + p_{\eB,4} r_4 \geq 30\% \\
    p_\eC &= p_{\eC,1} r_1 + p_{\eC,2} r_2 + p_{\eC,3} r_3 + p_{\eC,4} r_4 \geq 20\%
\end{align*}

O objetivo do programa seria então \textbf{minimizar} o custo $s$ por Kg da mistura a partir dos custos $c_i$ de cada composto, que é dado por:
\[
    s = c_1 r_1 + c_2 r_2 + c_3 r_3 + c_4 r_4
\]

Além disso, a mistura é feita apenas dos compostos $i$ que não podem ser negativos, então
\begin{gather*}
    r_1 + r_2 + r_3 + r_4 = 100\% \\
    0 \leq r_1, r_2, r_3, r_4
\end{gather*}

As últimas restrições são somente nas proporções dos compostos 1 e 2 na mistura: $r_1 \leq 30\%$ e $r_2 \leq 40\%$.

~

\begin{description}
    \reducemathskip{}
    \item[Variáveis:] $r_i \doteq$ proporção do composto $i = 1, 2, 3, 4$ na mistura

    \item[Função objetivo:] $\min s = 20 r_1 + 30 r_2 + 20 r_3 + 15 r_4$

    \item[Restrição (i):] (proporção de A)
    \[ 0.3 r_1 + 0.2 r_2 + 0.4 r_3 + 0.2 r_4 = 0.2 \]

    \item[Restrição (ii):] (proporção de B)
    \[ 0.2 r_1 + 0.6 r_2 + 0.3 r_3 + 0.4 r_4 \geq 0.3 \]

    \item[Restrição (iii):] (proporção de C)
    \[ 0.4 r_1 + 0.15 r_2 + 0.25 r_3 + 0.3 r_4 \geq 0.2 \]

    \item[Restrição (iv):] (limite de uso do composto 1)
    \[ r_1 \leq 0.3 \]

    \item[Restrição (v):] (limite de uso do composto 2)
    \[ r_2 \leq 0.4 \]

    \item[Proporção da mistura:] $r_1 + r_2 + r_3 + r_4 = 1$

    \item[Não-negatividade:] $r_1, r_2, r_3, r_4 \geq 0$
\end{description}
